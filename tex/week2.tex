\section*{Week 2 - Linear Response Theory}\label{sec:week2}
Victor will write more on the theory from week 2 during the fall break.
\textbf{Reading Material [GP]}
\begin{itemize}
    \item Section 7.\{2,3,4,5\} 
    \item Appendix 7\{B,C\}
\end{itemize}

\emph{Kristian's Appetiser:} Reading guidance: Sections 7.2 and 7.3 gives semi-classical description of static and dynamical screening in metals, respectively. Section 7.4 introduces some general facts about response functions (in particular the Kramers-Kronig relations) and states the full quantum mechanical expression for the dielectric function, Eq. (7.24). The derivation of this important expression is done in appendix 7B. Section 7.5 revisits static and dynamical screening in metals using the quantum expression for the dielectric function (the semi-classical treatments were given in 7.2 and 7.3, respectively). The results in this section are based on the quantum dielectric functional specialised to the case of a homogeneous electron gas where the wave functions are simply plane waves. In this special case, the dielectric function is known as the Lindhart function, named after the danish physicist Jens Lindhard, who obtained a closed analytical expression for the dielectric function of the homogeneous metal in 1954. - see appendix 7C. 