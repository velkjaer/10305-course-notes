\section{Week 8}
The retarded density-density response function can be used to obtain the longitudinal dielectric function of a material due to the relation between the induced potential and the induced electron density. Furthermore, it is useful when trying to identify \textit{neutral} excitations of the electron system. Such excitations include single-particle transition, plasmons and excitons. 

Conversely, charged excitations can be described by another correlation function, namely, the one-particle Green's function (GF). More precisely, charged excitations are excitations where the number of electrons in the electron system changes by $\pm 1$, e.g. photo-emission or STM.\footnote{Charged excitations do sometimes go under the name of addition/removal energies, anti-particle energies or one-electron energies.} 
The energies of charged excitations, $E_i \qty(N\pm1) - E_0\qty(N)$ are differences between exact many-body energy levels and thus completely well-defined. These types of excitations will from now on be denoted quasiparticles (QP).

\paragraph{The Retarded Green's function} is defined in real time and space as
\begin{equation}
    G\qty(x,x') = -i \theta\qty(t-t') \mel{N}{\{\fop{x},\fopd{x'}\}}{N}
\end{equation}
where $x = (r,t)$ and $\fop{x}=\expe{iHt}\fop{r} \expe{-iHt}$ is the annihilation operator in the Heisenberg picture.\footnote{In the following we assume that $\hat{H}$ is time-independent.}
Note that the two terms from the anti=